\documentclass[a4paper,14pt]{extarticle}
\usepackage[T2A]{fontenc}
\usepackage[utf8]{inputenc}
\usepackage[russian]{babel}

\usepackage[dvips]{graphicx}
\usepackage{color}
\usepackage{hyperref}
\usepackage{listings}
\usepackage{xcolor}

\usepackage{setspace}
\usepackage{indentfirst}
\usepackage{textcomp}
\usepackage{ifthen}
\usepackage{calc}
\usepackage{float} % Добавлен пакет для управления размещением float-объектов

\hypersetup{%
	unicode,%
	linkcolor=blue,
	colorlinks=true,%
	pdftitle={Курсовая работа по дисциплине "Фундаментальная информатика"},%
}

\thispagestyle{empty}

\setlength{\voffset}{-.8in}
\setlength{\hoffset}{-.75in}
\addtolength{\textheight}{1.6in}
\addtolength{\textwidth}{1.5in}

\onehalfspacing

% Настройки для цветных листингов
\lstset{
    language=SQL,
    basicstyle=\ttfamily\footnotesize,
    keywordstyle=\color{blue}\bfseries,
    commentstyle=\color{green!50!black},
    stringstyle=\color{red},
    showstringspaces=false,
    breaklines=true,
    frame=single,
    numbers=left,
    numberstyle=\tiny\color{gray},
    tabsize=2,
    captionpos=b,
    literate={а}{{\selectfont\char224}}1
             {б}{{\selectfont\char225}}1
             {в}{{\selectfont\char226}}1
             {г}{{\selectfont\char227}}1
             {д}{{\selectfont\char228}}1
             {е}{{\selectfont\char229}}1
             {ё}{{\"e}}1
             {ж}{{\selectfont\char230}}1
             {з}{{\selectfont\char231}}1
             {и}{{\selectfont\char232}}1
             {й}{{\selectfont\char233}}1
             {к}{{\selectfont\char234}}1
             {л}{{\selectfont\char235}}1
             {м}{{\selectfont\char236}}1
             {н}{{\selectfont\char237}}1
             {о}{{\selectfont\char238}}1
             {п}{{\selectfont\char239}}1
             {р}{{\selectfont\char240}}1
             {с}{{\selectfont\char241}}1
             {т}{{\selectfont\char242}}1
             {у}{{\selectfont\char243}}1
             {ф}{{\selectfont\char244}}1
             {х}{{\selectfont\char245}}1
             {ц}{{\selectfont\char246}}1
             {ч}{{\selectfont\char247}}1
             {ш}{{\selectfont\char248}}1
             {щ}{{\selectfont\char249}}1
             {ъ}{{\selectfont\char250}}1
             {ы}{{\selectfont\char251}}1
             {ь}{{\selectfont\char252}}1
             {э}{{\selectfont\char253}}1
             {Ю}{{\selectfont\char254}}1
             {я}{{\selectfont\char255}}1
             {А}{{\selectfont\char192}}1
             {Б}{{\selectfont\char193}}1
             {В}{{\selectfont\char194}}1
             {Г}{{\selectfont\char195}}1
             {Д}{{\selectfont\char196}}1
             {Е}{{\selectfont\char197}}1
             {Ё}{{\"E}}1
             {Ж}{{\selectfont\char198}}1
             {З}{{\selectfont\char199}}1
             {И}{{\selectfont\char200}}1
             {Й}{{\selectfont\char201}}1
             {К}{{\selectfont\char202}}1
             {Л}{{\selectfont\char203}}1
             {М}{{\selectfont\char204}}1
             {Н}{{\selectfont\char205}}1
             {О}{{\selectfont\char206}}1
             {П}{{\selectfont\char207}}1
             {Р}{{\selectfont\char208}}1
             {С}{{\selectfont\char209}}1
             {Т}{{\selectfont\char210}}1
             {У}{{\selectfont\char211}}1
             {Ф}{{\selectfont\char212}}1
             {Х}{{\selectfont\char213}}1
             {Ц}{{\selectfont\char214}}1
             {Ч}{{\selectfont\char215}}1
             {Ш}{{\selectfont\char216}}1
             {Щ}{{\selectfont\char217}}1
             {Ъ}{{\selectfont\char218}}1
             {Ы}{{\selectfont\char219}}1
             {Ь}{{\selectfont\char220}}1
             {Э}{{\selectfont\char221}}1
             {Ю}{{\selectfont\char222}}1
             {Я}{{\selectfont\char223}}1
}

% Поменяйте фамилию, имя и отчество в команде FIO
\newcommand{\FIO}{Молчанов Даниил Максимович}

% Поменяйте название программного продукта в команде SOFTWARE
\newcommand{\SOFTWARE}{Система управления библиотекой}

% Укажите номер группы (110 или 111)
\newcommand{\GROUP}{ИТПМ-124}

\begin{document}
\begin{center}
{\large	МИНИСТЕРСТВО НАУКИ И ВЫСШЕГО ОБРАЗОВАНИЯ РФ 
	
	\textbf{РГУ им. А. Н. Косыгина }
	
	\textbf{(Технологии. Дизайн. Искусство.)} 
	\bigskip
	
	Институт Информационных технологий и цифровой трансформации}
	
	\bigskip
	
	\textbf{Кафедра <<Искусственного интеллекта, \\ прикладной математики и программирования>>}
	\bigskip
	
	\vfill \textsc{\Large курсовой работа} \\
	{\large по дисциплине <<Базы данных и программирование>>}
	\bigskip

	на тему: Задачи управления базой данных  
\end{center}
\vspace*{1.5cm}

\hfill 
\begin{minipage}{.6\linewidth}\
\begin{tabular}{c}
\textbf{Выполнил:} студент группы \GROUP{} \\\hline \\[.3cm]
{\large \FIO} \\ \hline \scriptsize{(Фамилия, имя, отчество)}
\\[.3cm] \\ \hline
\scriptsize{(подпись)} \\[.3cm]
\textbf{Принял: }\hfill доцент кафедры ИИПМиП \\\hline 
\\[.3cm]

Мокряков Алексей Викторович
%\ifthenelse{\equal{\GROUP}{110}}{Карташов Эдуард Михайлович}{}%
%\ifthenelse{\equal{\GROUP}{111}}{Лисов Александр Александрович}{}\\\hline 
\scriptsize{(Фамилия, имя, отчество)}
\\[.3cm] \\ \hline
\scriptsize{(подпись)} 
\end{tabular}
\vspace*{1cm}
\end{minipage}		

\centerline{\textbf{Оценка}: \hspace*{8cm} \textbf{Дата}: \hspace*{2cm}}
\vspace*{1cm}

\centerline{Москва, 2025}

\newpage

\tableofcontents
\newpage

\section{Введение}
В данном разделе описывается цель курсовой работы, решаемые задачи и структура документа.

\section{Техническое задание}
Разработка системы управления библиотекой на основе реляционной базы данных PostgreSQL. Система должна обеспечивать:
\begin{itemize}
    \item Управление каталогом книг, авторов и издательств
    \item Учет читателей и их истории выдачи
    \item Управление физическими экземплярами книг с учетом их состояния
    \item Бронирование книг читателями с ограничениями
    \item Логирование операций в системе
    \item Формирование отчетов и представлений
    \item Проверку корректности операций (просрочки, доступность)
    \item Автоматическое обновление статусов экземпляров
    \item Поиск книг по различным критериям
\end{itemize}

\section{Проектирование базы данных}

\subsection{Создание базы данных}
\begin{lstlisting}[caption={Создание базы данных t01\_library}]
CREATE DATABASE t01_library;
\end{lstlisting}

\begin{figure}[H] % Изменено с [ht] на [H]
    \centering
    \includegraphics[width=0.8\textwidth]{1.png}
    \caption{Результат создания базы данных}
    \label{fig:db_creation}
\end{figure}

\subsection{Создание таблицы author}
\begin{lstlisting}[caption={Создание таблицы author}]
create table public.author(
id serial primary key,
first_name varchar(80) not null,
last_name varchar(80) not null
);
\end{lstlisting}

\subsection{Создание таблицы publishing\_house}
\begin{lstlisting}[caption={Создание таблицы publishing\_house}]
create table public.publishing_house(
id SERIAL primary key,
name varchar(80) not null,
city varchar(80) not null
);
\end{lstlisting}

\subsection{Создание таблицы book}
\begin{lstlisting}[caption={Создание таблицы book}]
create table public.book(
id serial primary key,
name_book varchar(80) not null,
author int references public.author(id),
publisher_code int references public.publishing_house(id),
edition int not null,
year_publication int,
circulation int
);
\end{lstlisting}

\subsection{Создание таблицы reader}
\begin{lstlisting}[caption={Создание таблицы reader}]
create type sex as enum ('male', 'female');

create table public.reader(
id serial primary key,
first_name varchar(80) not null,
last_name varchar(80) not null,
date_of_birth date,
gender sex,
date_of_regestration date
);
\end{lstlisting}

\subsection{Создание таблицы book\_instance}
\begin{lstlisting}[caption={Создание таблицы book\_instance}]
create type availability as enum ('in stock', 'issued', 'reserved');
create type state as enum ('excellent', 'good', 'satisfactory', 'dilapidated', 'lost');

create table public.book_instance(
id serial primary key,
book_information int references public.book(id),
state state, 
status availability,
location varchar(80) not null
);
\end{lstlisting}

\subsection{Создание таблицы issuance}
\begin{lstlisting}[caption={Создание таблицы issuance}]
create table public.issuance(
reader_id int references public.reader(id),
book_id int references public.book_instance(id),
date_time timestamp,
date_return_expected date not null,
date_return_fact date NULL
);
\end{lstlisting}

\subsection{Создание таблицы booking}
\begin{lstlisting}[caption={Создание таблицы booking}]
create table booking(
id serial primary key,
number_reader_ticket int references public.reader(id),
book_information int references public.book(id),
data_time timestamp,
min_state state
);
\end{lstlisting}

\begin{figure}[H] % Изменено с [ht] на [H]
    \centering
    \includegraphics[width=0.8\textwidth]{2.png} % имя вашего файла
    \caption{Результат создания таблиц}
    \label{fig:table_booking}
\end{figure}

\section{Реализация бизнес-логики}

\subsection{Управление авторами (CRUD операции)}
\begin{lstlisting}[caption={CRUD операции для таблицы author}]
INSERT INTO author (first_name, last_name) VALUES 
('Александр','Пушкин'),
('Михаил','Булгаков'),
('Лев','Толстой'),
('Федор','Достоевский'),
('Антон','Чехов'),
('Николай','Гоголь'),
('Иван','Тургенев'),
('Владимир','Набоков'),
('Сергей','Есенин'),
('Анна','Ахматова')
RETURNING id;

UPDATE author SET last_name='Ахматова-Горенко' WHERE first_name='Анна';

DELETE FROM author WHERE id IN (2, 5);
\end{lstlisting}

\begin{figure}[H] % Изменено с [ht] на [H]
    \centering
    \includegraphics[width=0.8\textwidth]{3.png}
    \caption{Результат операции INSERT}
    \label{fig:author_insert}
\end{figure}

\begin{figure}[H] % Изменено с [ht] на [H]
    \centering
    \includegraphics[width=0.8\textwidth]{4.png}
    \caption{Результат операции UPDATE}
    \label{fig:author_update}
\end{figure}

\begin{figure}[H] % Изменено с [ht] на [H]
    \centering
    \includegraphics[width=0.8\textwidth]{5.png}
    \caption{Результат операции DELETE}
    \label{fig:author_delete}
\end{figure}

\subsection{Управление издательствами}
\begin{lstlisting}[caption={CRUD операции для таблицы publishing\_house}]
INSERT INTO publishing_house (name, city) VALUES 
('Солнышко','Москва'),
('АСТ','Москва'),
('Эксмо','Санкт-Петербург'),
('Просвещение','Москва'),
('Росмэн','Казань'),
('Дрофа','Москва'),
('Фламинго','Санкт-Петербург'),
('Олма-Пресс','Новосибирск'),
('Азбука','Москва'),
('Амфора','Екатеринбург')
RETURNING id;

UPDATE publishing_house SET name='Просвещение+' WHERE id=4;

DELETE FROM publishing_house WHERE id IN (2, 4, 6);
\end{lstlisting}

\begin{figure}[H] % Изменено с [ht] на [H]
    \centering
    \includegraphics[width=0.8\textwidth]{6.png}
    \caption{Результат операций с издательствами}
    \label{fig:publishing_house_results}
\end{figure}

\begin{figure}[H] % Изменено с [ht] на [H]
    \centering
    \includegraphics[width=0.8\textwidth]{7.png}
    \caption{Дополнительный результат для издательств}
    \label{fig:publishing_house_extra1}
\end{figure}

\begin{figure}[H] % Изменено с [ht] на [H]
    \centering
    \includegraphics[width=0.8\textwidth]{8.png}
    \caption{Финальный результат для издательств}
    \label{fig:publishing_house_extra2}
\end{figure}

\subsection{Управление информацией о книгах}
\begin{lstlisting}[caption={CRUD операции для таблицы book}]
INSERT INTO book (name_book, author, publisher_code, edition, year_publication, circulation) VALUES
('Колобок', 1, 1, 1, 2012, 5000),
('Маша и медведь', 1, 3, 1, 2002, 5200),
('Мастер и Маргарита', 2, 2, 1, 1966, 10000),
('Война и мир', 3, 3, 1, 1869, 8000),
('Преступление и наказание', 4, 1, 2, 1866, 7000),
('Евгений Онегин', 1, 2, 3, 1833, 12000),
('Мертвые души', 6, 3, 1, 1842, 6000),
('Отцы и дети', 7, 1, 2, 1862, 5500),
('Лолита', 8, 2, 1, 1955, 3000),
('Анна Каренина', 3, 3, 2, 1877, 9000)
RETURNING id;

UPDATE book SET name_book='Маша и три медведя' WHERE id=2;

DELETE FROM book WHERE id IN (9,10);
\end{lstlisting}

\begin{figure}[H] % Изменено с [ht] на [H]
    \centering
    \includegraphics[width=0.8\textwidth]{9.png}
    \caption{Результат операций с книгами - часть 1}
    \label{fig:book_results1}
\end{figure}

\begin{figure}[H] % Изменено с [ht] на [H]
    \centering
    \includegraphics[width=0.8\textwidth]{10.png}
    \caption{Результат операций с книгами - часть 2}
    \label{fig:book_results2}
\end{figure}

\begin{figure}[H] % Изменено с [ht] на [H]
    \centering
    \includegraphics[width=0.8\textwidth]{11.png}
    \caption{Результат операций с книгами - часть 3}
    \label{fig:book_results3}
\end{figure}

\subsection{Управление читателями}
\begin{lstlisting}[caption={CRUD операции для таблицы reader}]
INSERT INTO reader (first_name, last_name, date_of_birth, gender, date_of_regestration) VALUES
('Даниил', 'Молчанов', '2006-10-25', 'male', CURRENT_DATE),
('Иван', 'Иванов', '1950-10-24', 'male', CURRENT_DATE),
('Алексей', 'Перов', '1980-09-21', 'male', CURRENT_DATE),
('Мария', 'Иванова', '2000-05-15', 'female', '2024-01-10'),
('Ольга', 'Петрова', '1995-12-01', 'female', CURRENT_DATE - INTERVAL '5 days'),
('Сергей', 'Сидоров', '1975-03-18', 'male', CURRENT_DATE - INTERVAL '10 days'),
('Екатерина', 'Кузнецова', '1988-07-30', 'female', CURRENT_DATE - INTERVAL '3 days'),
('Дмитрий', 'Васильев', '1992-11-11', 'male', '2024-01-15'),
('Анастасия', 'Попова', '1999-04-22', 'female', CURRENT_DATE),
('Михаил', 'Соколов', '1965-08-14', 'male', CURRENT_DATE - INTERVAL '7 days')
RETURNING id;

UPDATE reader SET date_of_regestration='2024-01-20' WHERE id=10;

DELETE FROM reader WHERE id IN (1,2,3);
\end{lstlisting}

\begin{figure}[H] % Изменено с [ht] на [H]
    \centering
    \includegraphics[width=0.8\textwidth]{12.png}
    \caption{Результат операций с читателями - часть 1}
    \label{fig:reader_results1}
\end{figure}

\begin{figure}[H] % Изменено с [ht] на [H]
    \centering
    \includegraphics[width=0.8\textwidth]{13.png}
    \caption{Результат операций с читателями - часть 2}
    \label{fig:reader_results2}
\end{figure}

\begin{figure}[H] % Изменено с [ht] на [H]
    \centering
    \includegraphics[width=0.8\textwidth]{14.png}
    \caption{Результат операций с читателями - часть 3}
    \label{fig:reader_results3}
\end{figure}

\subsection{Управление экземплярами книг}
\begin{lstlisting}[caption={CRUD операции для таблицы book\_instance}]
INSERT INTO book_instance (book_information, state, status, location) VALUES
(1, 'good', 'in stock', 'Стеллаж 1'),
(2, 'excellent', 'reserved', 'Стеллаж 2'),
(3, 'good', 'issued', 'Стеллаж 3'),
(4, 'good', 'in stock', 'Стеллаж 4'),
(5, 'satisfactory', 'in stock', 'Стеллаж 5'),
(6, 'excellent', 'in stock', 'Стеллаж 1'),
(7, 'good', 'reserved', 'Стеллаж 2'),
(8, 'satisfactory', 'issued', 'Стеллаж 3'),
(9, 'good', 'in stock', 'Стеллаж 4'),
(10, 'excellent', 'in stock', 'Стеллаж 5')
RETURNING id;

UPDATE book_instance SET status='issued' WHERE id=9;

DELETE FROM book_instance WHERE id IN (4);
\end{lstlisting}

\begin{figure}[H] % Изменено с [ht] на [H]
    \centering
    \includegraphics[width=0.8\textwidth]{15.png}
    \caption{Результат операций с экземплярами книг - часть 1}
    \label{fig:instance_results1}
\end{figure}

\begin{figure}[H] % Изменено с [ht] на [H]
    \centering
    \includegraphics[width=0.8\textwidth]{16.png}
    \caption{Результат операций с экземплярами книг - часть 2}
    \label{fig:instance_results2}
\end{figure}

\begin{figure}[H] % Изменено с [ht] на [H]
    \centering
    \includegraphics[width=0.8\textwidth]{17.png}
    \caption{Результат операций с экземплярами книг - часть 3}
    \label{fig:instance_results3}
\end{figure}

\subsection{Сценарий выдачи книги}
\begin{lstlisting}[caption={Процедура выдачи книги}]
insert into public.issuance (reader_id, book_id, date_time,date_return_expected,date_return_fact)
values(2,2,CURRENT_TIMESTAMP(0),CURRENT_DATE+ interval '14 days',null)
returning reader_id, book_id;

update public.book_instance
set status='issued'
where id=2;
\end{lstlisting}

\begin{figure}[H] % Изменено с [ht] на [H]
    \centering
    \includegraphics[width=0.8\textwidth]{18.png}
    \caption{Результат выдачи книги}
    \label{fig:issue_book}
\end{figure}

\begin{figure}[H] % Изменено с [ht] на [H]
    \centering
    \includegraphics[width=0.8\textwidth]{19.png}
    \caption{Результат обновления статуса книги}
    \label{fig:issue_book_update}
\end{figure}

\subsection{Сценарий возврата книги}
\begin{lstlisting}[caption={Процедура возврата книги}]
update public.book_instance
set status='in stock'
where id=2;

update public.issuance
set date_return_fact=CURRENT_DATE
where book_id=2 and reader_id=2;
\end{lstlisting}

\begin{figure}[H] % Изменено с [ht] на [H]
    \centering
    \includegraphics[width=0.8\textwidth]{20.png}
    \caption{Результат возврата книги - обновление статуса}
    \label{fig:return_book_status}
\end{figure}

\begin{figure}[H] % Изменено с [ht] на [H]
    \centering
    \includegraphics[width=0.8\textwidth]{21.png}
    \caption{Результат возврата книги - обновление даты возврата}
    \label{fig:return_book_date}
\end{figure}

\subsection{Представление выданных книг}
\begin{lstlisting}[caption={Представление issued\_books}]
insert into public.issuance (reader_id, book_id, date_time,date_return_expected,date_return_fact)
values
    (1,2,CURRENT_TIMESTAMP(0)- interval '3 days', CURRENT_DATE+ interval '11 days',null),
    (2,3,CURRENT_TIMESTAMP(0)- interval '8 days', CURRENT_DATE+ interval '2 days',null);

UPDATE public.book_instance
set status = 'issued'
WHERE id IN (2,3);
\end{lstlisting}

\begin{figure}[H] % Изменено с [ht] на [H]
    \centering
    \includegraphics[width=0.8\textwidth]{22.png}
    \caption{Результат добавления новых выдач}
    \label{fig:new_issues}
\end{figure}

\begin{figure}[H] % Изменено с [ht] на [H]
    \centering
    \includegraphics[width=0.8\textwidth]{23.png}
    \caption{Результат обновления статуса экземпляров}
    \label{fig:update_instance_status}
\end{figure}

\subsection{Представление просроченных выдач}
\begin{lstlisting}[caption={Представление overdue\_books}]
CREATE OR REPLACE VIEW public.issued_books_view AS
SELECT 
    r.last_name AS reader_last_name,
    r.first_name AS reader_first_name,
    a.last_name AS author_last_name,
    a.first_name AS author_first_name,
    b.name_book AS book_title,
    bi.state AS book_condition,
    i.date_time AS issue_date
FROM public.issuance i
JOIN public.reader r ON i.reader_id = r.id
JOIN public.book_instance bi ON i.book_id = bi.id
JOIN public.book b ON bi.book_information = b.id
JOIN public.author a ON b.author = a.id
WHERE i.date_return_fact IS NULL;

SELECT * FROM public.issued_books_view;
\end{lstlisting}

\begin{figure}[H] % Изменено с [ht] на [H]
    \centering
    \includegraphics[width=0.8\textwidth]{24.png}
    \caption{Результат представления issued\_books\_view}
    \label{fig:issued_books_view_result}
\end{figure}

\subsection{Модификация выдачи с проверкой просрочек}
\begin{lstlisting}[caption={Модифицированная процедура выдачи}]
CREATE OR REPLACE VIEW public.overdue_books_view AS
SELECT 
    r.last_name AS reader_last_name,
    r.first_name AS reader_first_name,
    a.last_name AS author_last_name,
    a.first_name AS author_first_name,
    b.name_book AS book_title,
    (CURRENT_DATE-i.date_return_expected) as overdue_days
FROM public.issuance i
JOIN public.reader r ON i.reader_id = r.id
JOIN public.book_instance bi ON i.book_id = bi.id
JOIN public.book b ON bi.book_information = b.id
JOIN public.author a ON b.author = a.id
WHERE i.date_return_fact IS null;

SELECT * from public.overdue_books_view;
\end{lstlisting}

\begin{figure}[H] % Изменено с [ht] на [H]
    \centering
    \includegraphics[width=0.8\textwidth]{25.png}
    \caption{Результат представления overdue\_books\_view}
    \label{fig:overdue_books_view_result}
\end{figure}

\subsection{Сценарий бронирования книги}
\begin{lstlisting}[caption={Процедура бронирования книги}]
CREATE OR REPLACE FUNCTION reserve_book(
    p_reader_id INTEGER,
    p_book_id INTEGER
) RETURNS BOOLEAN AS $$
DECLARE
    v_available_instance_id INTEGER;
    v_booking_count INTEGER;
    v_reservation_end_date DATE;
BEGIN
 
    SELECT COUNT(*) INTO v_booking_count 
    FROM public.booking 
    WHERE number_reader_ticket = p_reader_id 
    AND data_time + INTERVAL '3 days' > CURRENT_TIMESTAMP;
    
    IF v_booking_count >= 3 THEN
        RAISE EXCEPTION 'Превышено максимальное количество активных бронирований';
    END IF;
    
    SELECT bi.id INTO v_available_instance_id
    FROM public.book_instance bi
    WHERE bi.book_information = p_book_id 
    AND bi.status = 'in stock'
    AND bi.state >= 'satisfactory' 
    LIMIT 1;
    
    IF v_available_instance_id IS NULL THEN
        RAISE EXCEPTION 'Нет доступных экземпляров для бронирования';
    END IF;
    
    v_reservation_end_date := CURRENT_DATE + INTERVAL '3 days';
    
    INSERT INTO public.booking (
        number_reader_ticket, 
        book_information, 
        data_time,
        min_state
    ) VALUES (
        p_reader_id, 
        p_book_id, 
        CURRENT_TIMESTAMP,
        'satisfactory'
    );
    
    UPDATE public.book_instance 
    SET status = 'reserved' 
    WHERE id = v_available_instance_id;
    
    RETURN TRUE;
END;
$$ LANGUAGE plpgsql;

WITH available_books AS (
    SELECT bi.id as instance_id, b.id as book_id, b.name_book
    FROM book_instance bi
    JOIN book b ON bi.book_information = b.id
    WHERE bi.status = 'in stock'
    LIMIT 3
)
SELECT * FROM available_books;
\end{lstlisting}

\begin{figure}[H] % Изменено с [ht] на [H]
    \centering
    \includegraphics[width=0.8\textwidth]{26.png}
    \caption{Результат работы функции reserve\_book}
    \label{fig:reserve_book}
\end{figure}

\subsection{Сценарий отмены бронирования}
\begin{lstlisting}[caption={Процедура отмены бронирования}]
CREATE OR REPLACE FUNCTION cancel_reservation(
    p_reader_id INTEGER,
    p_book_id INTEGER
) RETURNS BOOLEAN AS $$
DECLARE
    v_booking_id INTEGER;
    v_instance_id INTEGER;
BEGIN
    SELECT b.id INTO v_booking_id
    FROM public.booking b
    WHERE b.number_reader_ticket = p_reader_id 
    AND b.book_information = p_book_id
    AND b.data_time + INTERVAL '3 days' > CURRENT_TIMESTAMP
    LIMIT 1;
    
    IF v_booking_id IS NULL THEN
        RAISE EXCEPTION 'Активное бронирование не найдено';
    END IF;
    
    SELECT bi.id INTO v_instance_id
    FROM public.book_instance bi
    WHERE bi.book_information = p_book_id 
    AND bi.status = 'reserved'
    LIMIT 1;
    
    DELETE FROM public.booking WHERE id = v_booking_id;
    
    IF v_instance_id IS NOT NULL THEN
        UPDATE public.book_instance 
        SET status = 'in stock' 
        WHERE id = v_instance_id;
    END IF;
    
    RETURN TRUE;
END;
$$ LANGUAGE plpgsql;

BEGIN;
SELECT reserve_book(4, 5);  -- Создать
SELECT cancel_reservation(4, 5);  -- Отменить
SELECT * FROM booking WHERE number_reader_ticket = 4;  -- Проверить (должно быть пусто)
SELECT * FROM book_instance WHERE book_information = 5; 
\end{lstlisting}

\begin{figure}[H] % Изменено с [ht] на [H]
    \centering
    \includegraphics[width=0.8\textwidth]{27.png}
    \caption{Результат работы функции cancel\_reservation - часть 1}
    \label{fig:cancel_reservation1}
\end{figure}

\begin{figure}[H] % Изменено с [ht] на [H]
    \centering
    \includegraphics[width=0.8\textwidth]{28.png}
    \caption{Результат работы функции cancel\_reservation - часть 2}
    \label{fig:cancel_reservation2}
\end{figure}

\begin{figure}[H] % Изменено с [ht] на [H]
    \centering
    \includegraphics[width=0.8\textwidth]{29.png}
    \caption{Результат работы функции cancel\_reservation - часть 3}
    \label{fig:cancel_reservation3}
\end{figure}

\subsection{Модификация выдачи с проверкой бронирований}
\begin{lstlisting}[caption={Проверка бронирований при выдаче}]
CREATE OR REPLACE FUNCTION issue_book_safe(
    p_reader_id INTEGER,
    p_instance_id INTEGER
) RETURNS BOOLEAN AS $$
DECLARE
    v_book_id INTEGER;
    v_overdue_count INTEGER;
    v_active_booking_count INTEGER;
    v_book_status availability;
BEGIN
    SELECT book_information, status 
    INTO v_book_id, v_book_status
    FROM public.book_instance 
    WHERE id = p_instance_id;
    
    IF v_book_id IS NULL THEN
        RAISE EXCEPTION 'Экземпляр книги не найден';
    END IF;
    
    SELECT COUNT(*) INTO v_overdue_count
    FROM public.issuance i
    WHERE i.reader_id = p_reader_id
      AND i.date_return_fact IS NULL
      AND i.date_return_expected < CURRENT_DATE;
    
    IF v_overdue_count > 0 THEN
        RAISE EXCEPTION 'У читателя есть просроченные книги';
    END IF;
    
    SELECT COUNT(*) INTO v_active_booking_count
    FROM public.booking b
    WHERE b.book_information = v_book_id
      AND b.number_reader_ticket != p_reader_id
      AND b.data_time + INTERVAL '3 days' > CURRENT_TIMESTAMP;
    
    IF v_active_booking_count > 0 THEN
        RAISE EXCEPTION 'Книга забронирована другим читателем';
    END IF;

    IF v_book_status NOT IN ('in stock', 'reserved') THEN
        RAISE EXCEPTION 'Книга недоступна для выдачи. Текущий статус: %', v_book_status;
    END IF;
    
    INSERT INTO public.issuance (
        reader_id, 
        book_id, 
        date_time,
        date_return_expected,
        date_return_fact
    ) VALUES (
        p_reader_id, 
        p_instance_id, 
        CURRENT_TIMESTAMP,
        CURRENT_DATE + INTERVAL '14 days',
        NULL
    );
    
    UPDATE public.book_instance
    SET status = 'issued'
    WHERE id = p_instance_id;
    
    DELETE FROM public.booking
    WHERE number_reader_ticket = p_reader_id
    AND book_information = v_book_id;
    
    RETURN TRUE;
END;
$$ LANGUAGE plpgsql;

SELECT issue_book_safe(4, 9);

SELECT 'Проверка выдачи:' as результат;
SELECT 
    r.first_name as читатель,
    b.name_book as книга,
    i.date_time as дата_выдачи,
    bi.status as статус_экземпляра
FROM issuance i
JOIN reader r ON i.reader_id = r.id
JOIN book_instance bi ON i.book_id = bi.id
JOIN book b ON bi.book_information = b.id;
\end{lstlisting}

\begin{figure}[H] % Изменено с [ht] на [H]
    \centering
    \includegraphics[width=0.8\textwidth]{30.png}
    \caption{Результат работы функции issue\_book\_safe}
    \label{fig:issue_book_safe}
\end{figure}

\subsection{Функция поиска местоположений книги}
\begin{lstlisting}[caption={Функция find\_book\_locations}]
CREATE OR REPLACE FUNCTION get_book_locations(p_book_id INTEGER)
RETURNS TABLE(
    location VARCHAR,
    book_state state,
    instance_count BIGINT,
    status availability
) AS $$
BEGIN
    RETURN QUERY
    SELECT 
        bi.location,
        bi.state,
        COUNT(*) as instance_count,
        bi.status
    FROM public.book_instance bi
    WHERE bi.book_information = p_book_id
    GROUP BY bi.location, bi.state, bi.status
    ORDER BY 
        CASE bi.state
            WHEN 'excellent' THEN 1
            WHEN 'good' THEN 2
            WHEN 'satisfactory' THEN 3
            WHEN 'dilapidated' THEN 4
            WHEN 'lost' THEN 5
            ELSE 6
        END,
        bi.location;
END;
$$ LANGUAGE plpgsql;

SELECT * FROM get_book_locations(8);
\end{lstlisting}

\begin{figure}[H] % Изменено с [ht] на [H]
    \centering
    \includegraphics[width=0.8\textwidth]{31.png}
    \caption{Результат работы функции get\_book\_locations}
    \label{fig:get_book_locations}
\end{figure}

\subsection{Представление доступных экземпляров}
\begin{lstlisting}[caption={Представление available\_instances}]
CREATE OR REPLACE VIEW public.available_books_view AS
SELECT 
    b.id as book_id,
    b.name_book,
    a.first_name as author_first_name,
    a.last_name as author_last_name,
    bi.state,
    COUNT(*) as available_count
FROM public.book b
JOIN public.author a ON b.author = a.id
JOIN public.book_instance bi ON bi.book_information = b.id
WHERE bi.status = 'in stock'
GROUP BY b.id, b.name_book, a.first_name, a.last_name, bi.state
ORDER BY b.name_book, 
    CASE bi.state
        WHEN 'excellent' THEN 1
        WHEN 'good' THEN 2
        WHEN 'satisfactory' THEN 3
        WHEN 'dilapidated' THEN 4
        ELSE 5
    END;

SELECT * FROM public.available_books_view;
\end{lstlisting}

\begin{figure}[H] % Изменено с [ht] на [H]
    \centering
    \includegraphics[width=0.8\textwidth]{32.png}
    \caption{Результат представления available\_books\_view}
    \label{fig:available_books_view}
\end{figure}

\subsection{Представление долгосрочных выдач}
\begin{lstlisting}[caption={Представление long\_term\_issuances}]
CREATE OR REPLACE VIEW public.overdue_books_year_view AS
SELECT 
    r.id as reader_id,
    r.last_name AS reader_last_name,
    r.first_name AS reader_first_name,
    a.last_name AS author_last_name,
    a.first_name AS author_first_name,
    b.name_book AS book_title,
    i.date_time AS issue_date,
    i.date_return_expected,
    (CURRENT_DATE - i.date_time::DATE) as days_issued
FROM public.issuance i
JOIN public.reader r ON i.reader_id = r.id
JOIN public.book_instance bi ON i.book_id = bi.id
JOIN public.book b ON bi.book_information = b.id
JOIN public.author a ON b.author = a.id
WHERE i.date_return_fact IS NULL
AND i.date_time < CURRENT_DATE - INTERVAL '1 year'
ORDER BY i.date_time;

SELECT * FROM public.overdue_books_year_view;
\end{lstlisting}

\begin{figure}[H] % Изменено с [ht] на [H]
    \centering
    \includegraphics[width=0.8\textwidth]{33.png}
    \caption{Результат представления overdue\_books\_year\_view}
    \label{fig:overdue_books_year_view}
\end{figure}

\subsection{Создание таблицы логов}
\begin{lstlisting}[caption={Создание таблицы logs}]
CREATE TABLE public.logs (
    id SERIAL PRIMARY KEY,
    log_date TIMESTAMP DEFAULT CURRENT_TIMESTAMP,
    table_name VARCHAR(100) NOT NULL,
    log_content TEXT NOT NULL
);
\end{lstlisting}

\begin{figure}[H] % Изменено с [ht] на [H]
    \centering
    \includegraphics[width=0.8\textwidth]{34.png}
    \caption{Результат создания таблицы logs}
    \label{fig:logs_table}
\end{figure}

\subsection{Реализация триггеров для логгирования}
\begin{lstlisting}[caption={Триггеры для автоматического логгирования}]
CREATE OR REPLACE FUNCTION log_changes()
RETURNS TRIGGER AS $$
DECLARE
    v_operation TEXT;
    v_content TEXT;
BEGIN
    IF TG_OP = 'INSERT' THEN
        v_operation := 'INSERT';
        v_content := 'Добавлена запись: ' || row_to_json(NEW);
    ELSIF TG_OP = 'UPDATE' THEN
        v_operation := 'UPDATE';
        v_content := 'Изменена запись. Старые данные: ' || row_to_json(OLD) || 
                    ' Новые данные: ' || row_to_json(NEW);
    ELSIF TG_OP = 'DELETE' THEN
        v_operation := 'DELETE';
        v_content := 'Удалена запись: ' || row_to_json(OLD);
    END IF;
    
    INSERT INTO public.logs (table_name, log_content)
    VALUES (TG_TABLE_NAME, v_operation || ' - ' || v_content);
    
    IF TG_OP = 'DELETE' THEN
        RETURN OLD;
    ELSE
        RETURN NEW;
    END IF;
END;
$$ LANGUAGE plpgsql;

CREATE TRIGGER trg_log_author
    AFTER INSERT OR UPDATE OR DELETE ON public.author
    FOR EACH ROW EXECUTE FUNCTION log_changes();

CREATE TRIGGER trg_log_book
    AFTER INSERT OR UPDATE OR DELETE ON public.book
    FOR EACH ROW EXECUTE FUNCTION log_changes();

CREATE TRIGGER trg_log_reader
    AFTER INSERT OR UPDATE OR DELETE ON public.reader
    FOR EACH ROW EXECUTE FUNCTION log_changes();

CREATE TRIGGER trg_log_book_instance
    AFTER INSERT OR UPDATE OR DELETE ON public.book_instance
    FOR EACH ROW EXECUTE FUNCTION log_changes();

CREATE TRIGGER trg_log_issuance
    AFTER INSERT OR UPDATE OR DELETE ON public.issuance
    FOR EACH ROW EXECUTE FUNCTION log_changes();

CREATE TRIGGER trg_log_booking
    AFTER INSERT OR UPDATE OR DELETE ON public.booking
    FOR EACH ROW EXECUTE FUNCTION log_changes();

-- Бронирование книги
SELECT reserve_book(4, 9);

-- Отмена бронирования
SELECT cancel_reservation(4, 1);

-- Выдача книги
SELECT issue_book(4, 1);

-- Получение информации о местоположениях книги
SELECT * FROM get_book_locations(8);

-- Просмотр доступных книг
SELECT * FROM available_books_view;

-- Просмотр долгосрочных просрочек
SELECT * FROM overdue_books_year_view;

-- Просмотр логов
SELECT * FROM logs ORDER BY log_date DESC;
\end{lstlisting}

\begin{figure}[H] % Изменено с [ht] на [H]
    \centering
    \includegraphics[width=0.8\textwidth]{35.png}
    \caption{Результат создания триггеров - часть 1}
    \label{fig:triggers1}
\end{figure}

\begin{figure}[H] % Изменено с [ht] на [H]
    \centering
    \includegraphics[width=0.8\textwidth]{36.png}
    \caption{Результат создания триггеров - часть 2}
    \label{fig:triggers2}
\end{figure}

\begin{figure}[H] % Изменено с [ht] на [H]
    \centering
    \includegraphics[width=0.8\textwidth]{37.png}
    \caption{Результат создания триггеров - часть 3}
    \label{fig:triggers3}
\end{figure}

\begin{figure}[H] % Изменено с [ht] на [H]
    \centering
    \includegraphics[width=0.8\textwidth]{38.png}
    \caption{Результат создания триггеров - часть 4}
    \label{fig:triggers4}
\end{figure}

\begin{figure}[H] % Изменено с [ht] на [H]
    \centering
    \includegraphics[width=0.8\textwidth]{39.png}
    \caption{Результат создания триггеров - часть 5}
    \label{fig:triggers5}
\end{figure}

\begin{figure}[H] % Изменено с [ht] на [H]
    \centering
    \includegraphics[width=0.8\textwidth]{40.png}
    \caption{Результат создания триггеров - часть 6}
    \label{fig:triggers6}
\end{figure}

\begin{figure}[H] % Изменено с [ht] на [H]
    \centering
    \includegraphics[width=0.8\textwidth]{41.png}
    \caption{Результат создания триггеров - часть 7}
    \label{fig:triggers7}
\end{figure}

\section{Заключение}
В результате выполнения курсовой работы была создана полнофункциональная система управления библиотекой, демонстрирующая практическое применение знаний в области проектирования баз данных, SQL-программирования и разработки бизнес-логики. Система соответствует современным требованиям к информационным системам и может служить основой для дальнейшего развития и внедрения в реальные библиотечные процессы. Работа подтвердила важность правильного проектирования структуры базы данных и эффективного использования возможностей СУБД для создания надежных и производительных информационных систем. 
\section{Список использованных источников}
\begin{enumerate}
    \item Грофф Дж., Вайнберг П. SQL: полное руководство. – 3-е изд. – М.: Вильямс, 2019. – 960 с.
    \item PostgreSQL: мощь открытых СУБД / Под ред. О. В. Самойленко. – СПб.: БХВ-Петербург, 2020. – 576 с.
    \item Таненбаум Э., ван Стеен М. Распределенные системы. Принципы и парадигмы. – 2-е изд. – СПб.: Питер, 2021. – 1104 с.
    \item Базы данных: проектирование, реализация, сопровождение. Теория и практика / К. Дж. Дейт. – М.: Вильямс, 2022. – 1440 с.
    \item SQL и реляционная теория. Как грамотно писать код на SQL / К. Дж. Дейт. – М.: Символ-Плюс, 2018. – 480 с.
\end{enumerate}

\end{document}